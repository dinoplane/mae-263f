%%%%%%%%%%%%%%%%%%%%%%%%%%%%%%%%%%%%%%%%%%%%%%%%%%%%%%%%%%%%%%%%%%%%%%%%%%%%%%%%
%2345678901234567890123456789012345678901234567890123456789012345678901234567890
%        1         2         3         4         5         6         7         8

\documentclass[letterpaper, 10 pt, conference]{ieeeconf}  % Comment this line out if you need a4paper
\usepackage{amstext}
%\documentclass[a4paper, 10pt, conference]{ieeeconf}      % Use this line for a4 paper

\IEEEoverridecommandlockouts                              % This command is only needed if
                                                          % you want to use the \thanks command

\overrideIEEEmargins                                      % Needed to meet printer requirements.

%In case you encounter the following error:
%Error 1010 The PDF file may be corrupt (unable to open PDF file) OR
%Error 1000 An error occurred while parsing a contents stream. Unable to analyze the PDF file.
%This is a known problem with pdfLaTeX conversion filter. The file cannot be opened with acrobat reader
%Please use one of the alternatives below to circumvent this error by uncommenting one or the other
%\pdfobjcompresslevel=0
%\pdfminorversion=4

% See the \addtolength command later in the file to balance the column lengths
% on the last page of the document

% The following packages can be found on http:\\www.ctan.org
%\usepackage{graphics} % for pdf, bitmapped graphics files
%\usepackage{epsfig} % for postscript graphics files
%\usepackage{mathptmx} % assumes new font selection scheme installed
%\usepackage{times} % assumes new font selection scheme installed
%\usepackage{amsmath} % assumes amsmath package installed
%\usepackage{amssymb}  % assumes amsmath package installed

\title{\LARGE \bf
Homework 1 Report
}


\author{Arrian Chi% <-this % stops a space
}


\begin{document}



\maketitle
\thispagestyle{empty}
\pagestyle{empty}


%%%%%%%%%%%%%%%%%%%%%%%%%%%%%%%%%%%%%%%%%%%%%%%%%%%%%%%%%%%%%%%%%%%%%%%%%%%%%%%%
\begin{abstract}

In this homework, I have simulated 3 different systems: a system of 3 metal spheres connected by beams falling in a viscous liquid, the same system but with N spheres, and a bending elastic beam subject to a single point load. The findings in each of these systems is discussed in the following sections.

\end{abstract}


%%%%%%%%%%%%%%%%%%%%%%%%%%%%%%%%%%%%%%%%%%%%%%%%%%%%%%%%%%%%%%%%%%%%%%%%%%%%%%%%
\section{4.1 Rigid Spheres and Elastic Beam Falling in Viscous Flow}

In this problem, we consider 3 falling metal spheres on an elastic beam that is falling under gravity. The constants are shown in the following table:

\begin{table}[h]
\caption{Variables Involved in the System of 3 Metal Spheres}
\label{table_variables}
\begin{center}
\begin{tabular}{|c|c|c|}
\hline
Variable & Description & Sample Value \\
\hline
& & \\ $\mu$ & Viscosity of the fluid & 1000 Pa-s \\
& & \\ $\rho_{ \text{fluid} }$ & Density of the fluid & 1000 kg/m\(^3\) \\
& & \\ $\rho_{\text{metal}}$ & Density of the spheres & 7000 kg/m\(^3\) \\
& & \\ $R_1, R_2, R_3$ & Radius of each sphere & 0.005 m, 0.025 m, \\ & & 0.005 m \\
& & \\ $l$ & Length of the beam & 0.10 m \\
& & \\ $r_0$ & Cross-sectional radius  & 0.001 m \\ & of the beam & \\
& & \\ $E$ & Young's modulus & $1.0 \times * 10^{9}$ Pa \\ & of the beam & \\
& & \\ $EA$ & Stretching stiffness & $E\pi r_0^2$  \\ & of the beam & \\
& & \\ $EI$ & Bending stiffness & $E\pi r_0^4 / 4$  \\ & of the beam & \\
& & \\ $\Delta l$ & Half-length of & 0.05 m \\ & the beam & \\

& & \\ $x_1, x_2, x_3$ & x position of each & initial at 0, $\Delta l$, $2\Delta l$\\ & sphere & \\
& & \\ $y_1, y_2, y_3$ & y position of each & all initial at 0\\ & sphere & \\
& & \\ $m_1, m_2, m_3$ & Mass of each sphere & $\frac{4}{3}\pi R_i^3 \rho_{\text{metal}}$\\
& & \\ $C_i$ & Damping coefficient& $6 \pi \mu R_i$ \\ &  due to fluid & \\
& & \\ $g$ & Acceleration due to  & 9.81 m/s\(^2\) \\ & gravity & \\
& & \\ $W_i$ & Weight of each & $\frac{4}{3} \pi R_i^3(\rho_{\text{metal}}-\rho_{\text{fluid}}g)$ \\ & sphere & \\
\hline
\end{tabular}
\end{center}
\end{table}

It should also be noted that the elastic energy of the beam is given by the following equation:
\begin{center}
        $E^{\text{elastic}} = E^S_1 + E^S_2 + E^b$
\end{center}
where $E^S_1$ and $E^S_2$ are the stretching energies of the beam on the left and right sides of the beam, respectively, and $E^b$ is the bending energy of the beam. The stretching energy of the beam is given by the following equation:
\begin{center}
        $E^S_1 = \frac{1}{2}EA\left( 1 - \frac{\sqrt{(x_2 - x_1)^2 + (y_2 - y_1)^2 }}{\Delta l}\right)^2 \Delta l$ \\



        $E^S_2 = \frac{1}{2}EA\left( 1 - \frac{\sqrt{(x_3 - x_2)^2 + (y_3 - y_2)^2 }}{\Delta l}\right)^2 \Delta l$ \\



        $E^b = \frac{1}{2} \frac{EI}{\Delta l}\left( 2 \tan{\frac{\theta}{2}} \right)^2$
\end{center}




Using these variables, we may write our equations of motion for each sphere in the system:

(TODO: Write the 6 equations we used)

And the discretized equations of motion for the beam are given by the following equations:

(TODO: write the discretized version of the equations we used)

In the explicit method, I replace the velocity term used in the damping force with the old velocity at $t_k$. This makes the integration method less coupled and easier to work with.

(TODO: just copy the equations)

In the implicit method, I use the following Jacobian matrix to input into the Newton-Raphson method:

(TODO: insert how we calculate the Jacobian matrix here)

Now we may answer the questions posed in the assignment:

\subsection{Q1: Shape of the structure at t = {0, 0.01, 0.05, 0.10, 1.0, 10.0}}

% \begin{figure}[h]
%         \caption{Example of a parametric plot ($\sin (x), \cos(x), x$)}
%         \centering
%         \includegraphics[width=0.5\textwidth]{spiral}
% \end{figure}

(TODO: insert figures)

\subsection{Q2: Position and Velocity Plot of $R_2$  as a function of time(TODO: $t$)}

(TODO: insert figures)

\subsection{ Q2: Terminal velocity of the system}

The terminal velocity of the system (all 3 spheres) is -0.005926 m/s. We find this by letting the simulation run as long as possible and obtaining the final velocity of the system. (and examining that the graph of the velocity stays stagnant at that velocity).

(TODO: need to examine the velocity plots of $R_1$ and $R_3$)

\subsection{ Q3: What happens to the turning angle if all radii are the same? }

If all the radii of the spheres are the same the turning angle will remain 0. Gravity acts on each part of the beam with equal force, so each sphere has the same acceleration and the same velocity. Because they all started at the same y, their displacements will also remain the same. So the turning angle remains what it was in the initial condition, which is 0 degrees. The final shape and turning angle graph is shown in (TODO: Figure)

(TODO: calculate turning angle and look at the shape the structure)

\subsection{ Q4: What happens when we change the time step, specifically for the explicit simulation?}
At larger time steps, the explicit simulation has a harder time converging. In the simulation, when $\Delta t = 1.0 \times  10^2$, the simulation warns me of scalar overflow errors. The resultant graph is shown in Figure (TODO: insert number here). This behavior emerges from the fact that our simulation relies on discretized formulas to compute an approximation. These discretizations assume a small time step, but the magnitude of its granularity is found with trial and error. In our explicit implementation, we use the time step directly without calculating margins of error (there is no way to do so), whereas in the implicit implementation, we provide a tolerance that is used to make sure the values are in the neighborhood of the ground truth (measuring how close we are to our last guess). This means in our explicit implementation, the variables in our computation accumulate error at every iteration, eventually causing values to diverge from their analytical truths. This is why explicit simulations benefit from having small time steps, at the cost of a higher computation time (due to more iterations).

I would like to note here that I believe that another reason the explicit simulation fails at smaller time steps is because we are solving for the velocity (and the velocity is not )


\section{4.2 Generalized Case of Elastic Beam Falling in Viscous Flow}

(TODO Opt: make a explicit version)

In the general case of the elastic beam falling, we have $N = 21$ spheres. We use the same variables as in the previous case, but some variables are calculated
\subsection{Q1: Plot of position and velocity of middle node as a function of time}

\subsection{Q2: Final shape of deformed beam}

\subsection{Q3: What is the significance of spatial discretization and temporal discretization? }

(TODO: Include plots of terminal velocity vs number of nodes and terminal velocity vs time step size)

vs nodes -0.005833482666667278


\section{4.3 Elastic Beam Bending}

\subsection{Q1: Plot maximum vertical displacement of beam as a function of time. Compare it with the theoretical prediction from the Euler beam theory.}

(TODO: Plot the max vertical displacement and check if $y_{\text{max}}$ is a steady value. Determine if the value is accurate or not.)

\subsection{Q2: What is the benefit of the simulation over the predictions of the Euler Beam Theory?}

(TODO: Use simulation on $P = 20000$. Note that the Euler Beam Theory is only accurate for small deflections.)

(TODO opt: make a plot of P vs $y_{\text{max}}$ using data from the simulation and the beam theory and find the P where the solutions diverge.)


   \begin{figure}[thpb]
      \centering
      \framebox{\parbox{3in}{We suggest that you use a text box to insert a graphic (which is ideally a 300 dpi TIFF or EPS file, with all fonts embedded) because, in an document, this method is somewhat more stable than directly inserting a picture.
}}
      %\includegraphics[scale=1.0]{figurefile}
      \caption{Inductance of oscillation winding on amorphous
       magnetic core versus DC bias magnetic field}
      \label{figurelabel}
   \end{figure}


\section{CONCLUSIONS}

A conclusion section is not required. Although a conclusion may review the main points of the paper, do not replicate the abstract as the conclusion. A conclusion might elaborate on the importance of the work or suggest applications and extensions.

\addtolength{\textheight}{-12cm}   % This command serves to balance the column lengths
                                  % on the last page of the document manually. It shortens
                                  % the textheight of the last page by a suitable amount.
                                  % This command does not take effect until the next page
                                  % so it should come on the page before the last. Make
                                  % sure that you do not shorten the textheight too much.

%%%%%%%%%%%%%%%%%%%%%%%%%%%%%%%%%%%%%%%%%%%%%%%%%%%%%%%%%%%%%%%%%%%%%%%%%%%%%%%%



%%%%%%%%%%%%%%%%%%%%%%%%%%%%%%%%%%%%%%%%%%%%%%%%%%%%%%%%%%%%%%%%%%%%%%%%%%%%%%%%



\end{document}
